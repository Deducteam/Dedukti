\documentclass[12pt,aspectratio=169]{beamer}
\usepackage[francais]{babel}
\usepackage[utf8]{inputenc}
\usepackage[T1]{fontenc}
\usepackage{lmodern}
\usepackage{../themes/logictheme}
\usepackage{wasysym}
\usefonttheme[onlymath]{serif}
\usepackage{relsize}
\usepackage{alltt}
\usepackage{fancybox}
\usepackage{listings}

\lstset{ %
  basicstyle=\ttfamily,
  keepspaces=true,                 % keeps spaces in text, useful for keeping indentation of code (possibly needs columns=flexible)
  deletekeywords={char*},
  showtabs=true,                  % show tabs within strings adding
                                % particular underscores
  escapechar=œ
}
\newcommand\lst[1]{{\lstinline{#1}}}

\DefineNamedColor{named}{mygray}{RGB}{58,58,58}
\usecolortheme[named=mygray]{structure}
\DefineNamedColor{named}{lblue}{RGB}{17,152,29}
\setbeamercolor{palette secondary}{fg=lblue}


\usepackage{etex}
\usepackage{latexsym}
\usepackage{bussproofs}
\usepackage[all,pdf]{xy}
\usepackage{alltt}
\input{macros}
\input{../Ringberg/calseq}
%\input{../Ringberg/natded}
%\input{../Ringberg/resolution}
\newcommand{\circlize}[1]{%
%  \usebeamerfont*{subitem}%
  \footnotesize
  \usebeamercolor[bg]{subitem projected}%
  \begin{pgfpicture}{-1ex}{0ex}{1ex}{2ex}
    \pgfpathcircle{\pgfpoint{0pt}{.6ex}}{1.2ex}
    \pgfusepath{fill}
    \pgftext[base]{\color{fg}\textbf{#1}}
  \end{pgfpicture}%
}

\title{Exprimer ses théories en Dedukti,\\ le vérificateur de preuves universel}
\subtitle{JFLA 2017}
\author[Guillaume Burel]{Guillaume Burel}
\date[2017-01-06/07]{vendredi 6 et samedi 7 janvier 2017}
\institute{ENSIIE/Samovar}
\logo{\includegraphics[height=.07\textheight]{../logos/ensiie.png}\includegraphics[height=.07\textheight]{../logos/samovar.png}
}

\begin{document}

\begin{frame}
  \titlepage
\end{frame}

\begin{frame}
  \frametitle{Dedukti}
  \begin{center}
    \colorbox{mygray}{\includegraphics{dedukti_logo.png}}
  \end{center}
  Vérificateur de preuve pour le $\lambda\Pi$-calcul modulo théorie
  \bigskip
  
  \url{http://dedukti.gforge.inria.fr/}
  \bigskip
  
  \url{http://dedukti.gforge.inria.fr/jfla2017/}

  
\end{frame}

\begin{frame}
  \frametitle{Dedukti: a Logical Framework based on the $\lambda\Pi$-Calculus Modulo Theory}


 Ali Assaf, Guillaume Burel,
Rapha\"el Cauderlier, David Delahaye, Gilles Dowek, Catherine Dubois,
Fr\'ed\'eric Gilbert, Pierre Halmagrand, Olivier Hermant et Ronan
Saillard.
\medskip

\url{http://www.lsv.ens-cachan.fr/~dowek/Publi/expressing.pdf}
\end{frame}

\begin{frame}
  \frametitle{Sommaire}
\tableofcontents   
\end{frame}

\section{Introduction}

\begin{frame}
    \begin{tabular}[b]{@{}c@{}}
\includegraphics[width=.25\textwidth]{../Auditions/geom}\\[1em]~
\end{tabular}\hfill
\includegraphics[width=.33\textwidth]{../from_picard/PhD/a380}  
\hfill  \begin{tabular}[b]{@{}c@{}}
\includegraphics[width=.2\textwidth]{../Auditions/securite}\\[1.5em]~
\end{tabular}  \bigskip

Les démonstrations sont recherchées dans des \textbf{théories}:
\begin{itemize}
\item géométrie, arithmétique
\item structures de données (listes chaînées, tableaux, \dots)
\item propriétés du chiffrement
\item \dots
\end{itemize}
\end{frame}

\begin{frame}
  \frametitle{Un cadre unique}
  Un seul cadre logique pour toutes les théories
  \begin{itemize}
  \item partage des connecteurs ($\wedge$, $\forall$, \dots)
  \item partage des notions de preuve, de modèle
    \begin{itemize}
    \item théorème généraux comme complétude prouvés une fois pour
      toute
    \end{itemize}
    \medskip

  \item Notions d'inclusion entre théories (exemple ZF $\subset$ ZFC)
    \medskip

  \item interopérabilité
    $\theo$ dans $\cal L$\qquad$\theo'$ dans $\cal L'$\qquad
    $A\in \cal L\cap \cal L'$
    \medskip
    
    $\theo \vdash A\imp B$\qquad $\theo'\vdash A$

    \qquad $\theo\cup\theo'\vdash B$
  \end{itemize}
\end{frame}

\begin{frame}
  \frametitle{Quel cadre utiliser ?}

  Logique du premier ordre ?
  \begin{itemize}
  \item Pour définir une théorie :
    \begin{itemize}
    \item symboles de fonction et de prédicat
    \item axiomes
    \end{itemize}
  \end{itemize}
  \medskip
  
  Mais pas unique en pratique :
  \begin{itemize}
  \item théorie simple des types (HOL)
  \item calcul des constructions inductives
  \end{itemize}
\end{frame}

\begin{frame}
  \frametitle{Inconvénient}
  Pas de possibilité de passer d'un prouveur à un autre
  \begin{itemize}
  \item CompCert en Isabelle/HOL ?
  \item seL4 en Coq ?
  \end{itemize}
  \bigskip
  
  Combinaison avec des outils de preuve automatique ?
  \begin{itemize}
  \item éviter la reconstruction de preuve
  \end{itemize}
\end{frame}

\begin{frame}
  \frametitle{Pourquoi ?}
  Manque en logique du premier ordre :
  \begin{enumerate}
  \item Possibilité de définir des lieurs
    \begin{itemize}
    \item exemple $x\mapsto t$
    \end{itemize}
  \item Pas de correspondance preuve/programme --- formule/type
  \item Pas de prise en compte du calcul
    \begin{itemize}
    \item simulé par des étapes de déduction
    \end{itemize}
  \item Pas de notion universelle de coupure
    \begin{itemize}
    \item doit être définie pour chaque théorie
    \end{itemize}
  \item Cadres différents pour logiques classique et intuitionniste
  \end{enumerate}
\end{frame}

\begin{frame}
  \frametitle{Résoudre problèmes 1 et 2}
  Logical Framework LF a.k.a. $\lambda P$ a.k.a. $\lambda\Pi$ [Harper~Honsell~Plotkin~1993]
  \medskip
  
  Extension
  \begin{itemize}
  \item du $\lambda$-calcul simplement typé avec des types
    dépendants
  \item de la logique des prédicats (minimale) avec des termes de preuve
  \end{itemize}
\end{frame}

\begin{frame}
  \frametitle{Résoudre problèmes 3 et 4}
  Déduction modulo théorie [Dowek~Hardin~Kirchner~2003]

Règles d'inférence appliquées modulo une congruence $\equiv$
\bigskip

Système de réécriture sur les termes \alert{et les propositions}
\medskip

Exemple: $X\in \mathcal P(Y) \rightarrow \forall Z.~Z\in X\imp Z\in Y$
\begin{prooftree}
  \AxiomSC$Z\in a\fCenter Z\in a$
  \ImpR$\fCenter Z\in a\imp Z\in a$
  \AllR[\scriptsize$a\in\mathcal P(a)\equiv \forall Z.~Z\in a\imp Z\in a$]$\fCenter a \in \mathcal P(a)$
\end{prooftree}
\end{frame}

\begin{frame}
  \frametitle{Résoudre 1, 2, 3 et 4}
  Combiner LF avec la déduction modulo théorie :
  \begin{itemize}
  \item $\lambda\Pi$-calcul modulo théorie
  \end{itemize}
  \bigskip
  
  Extension de la règle de conversion 
  \begin{itemize}
  \item sans modulo : uniquement $\beta$
  \item avec : $\beta$ + système de réécriture 
  \end{itemize}
\end{frame}

\begin{frame}
  \frametitle{Problème 5}
  Un connecteur $\triangleright$ pour passer d'atome à formule
  \medskip
  
  Deux versions des connecteurs : une constructive, une classique
\end{frame}

\begin{frame}
  \frametitle{Dedukti}
  $\lambda\Pi$-calcul modulo théorie pas seulement expressif en théorie
  \medskip

  Implémentation efficace:
  \begin{itemize}
  \item Dedukti
  \end{itemize}
  vérificateur de preuve du $\lambda\Pi$-calcul modulo théorie
  \bigskip

  M. Boespflug 2008 $\leadsto$ \dots{} $\leadsto$ Ronan Saillard 2015
  
\end{frame}

\begin{frame}
  \frametitle{Démo}
  
\end{frame}

\begin{frame}
  \frametitle{Outils autour de Dedukti}

  Traducteurs de preuves vers Dedukti
  \begin{itemize}
  \item HOL
  \item CIC
  \end{itemize}
  \bigskip
  
  Backend vers Dedukti
  \begin{itemize}
  \item prouveurs automatiques iProverModulo, Zenon modulo
  \item focalize
  \end{itemize}
\end{frame}

\begin{frame}
  \frametitle{Panorama actuel}
  \includegraphics[scale=1]{to_dedukti.pdf}
  
  En cours : Coq, PVS, VeriT, \dots
\end{frame}

\begin{frame}
  \frametitle{Plongement profond vs. plongement superficiel}
  Superficiel : réutilisation des fonctionnalités du langage cible :
\begin{itemize}
\item contextes
\item connecteurs et lieur
\item calcul
\item \dots
\end{itemize}
\bigskip

Intérêt des plongements superficiels
\begin{itemize}
\item traductions plus légères
\item plus efficace
\item rend l'interopérabilité plus simple
\end{itemize}

\end{frame}

\section{Le $\lambda\Pi$-calcul modulo théorie}



\begin{frame}
  \frametitle{Le $\lambda\Pi$-calcul}
  $t,u ::= Type~|~Kind~|~x~|~\lambda x:t.~u~|~t~u~|~\Pi x:t.~u$
  \bigskip
  
  niveaux:
  \begin{tabular}[t]{ll}
    $Kind$\\
    $Type$\qquad $\Pi x:nat.~Type$&types des (familles de) types\\
    $nat\qquad \Pi x:nat.~vect~x$&(familles) de types\\
    0\qquad $S$ \qquad$nil$\qquad$cons$ & objets
  \end{tabular}
  
\end{frame}

\begin{frame}[fragile]
  \frametitle{Types dépendants}
 \lstinline!x : nat -> nat! \textcolor{green}{\checkmark}
 \medskip
 
 \lstinline!x : nat -> vect x! \textcolor{green}{\checkmark}
\medskip
 
 \lstinline!x : nat -> Type! \textcolor{green}{\checkmark}
\medskip
 
 \lstinline!A : Type -> list A! \textcolor{red}{X}
\medskip
 
 \lstinline!A : Type -> Type! \textcolor{red}{X}
\end{frame}


\begin{frame}
  \frametitle{$\lambda\Pi$-calcul modulo théorie}
  En plus des déclarations de variables

  On va rajouter des règles de réécriture dans les contextes

  $$l\rr^\Delta r$$
  
  $\Delta$ : contexte local à la règle, ne contient que des
  déclarations de variables objets
  \bigskip

  $$\irule{\Gamma \vdash A:Type~~\Gamma \vdash B:Type~~\Gamma \vdash t:A}
        {\Gamma \vdash t:B}
        {A \eqbg B}$$
On peut réécrire avec les règles dans $\Gamma$
\end{frame}

\begin{frame}
  \frametitle{Bonnes hypothèses}
  Quelles hypothèses pour:
  \begin{itemize}
  \item réduction du sujet
  \item décidabilité de la vérification de type ?
  \end{itemize}
  \bigskip

  \begin{definition}[Bon typage d'une règle]
    $l\rr^\Delta r$ est bien typé si :
    
    Pour toute substitution
    $\sigma$ des variables de $\Delta$,

    si $\Gamma\vdash \sigma l:T$ alors $\Gamma\vdash \sigma r:T$.
  \end{definition}

  \begin{definition}[Compatibilité avec le produit]
    si $\Pi x:A_1~B_1 \eqbg \Pi x:A_2~B_2 \mbox{ then } A_1 \eqbg
A_2~\text{and}~B_1 \eqbg B_2$
  \end{definition}

\end{frame}

\begin{frame}
  \frametitle{Implications}
  \includegraphics{props.pdf}
\end{frame}

\begin{frame}[fragile]
  \frametitle{Bon typage des règles}
$l\rr^\Delta r$ bien typé si $\Gamma,\Delta\vdash l : T$ et
  $\Gamma,\Delta \vdash r : T$
\medskip
  
  mais ne marche pas pour
  $tail~n~(cons~m~a~v)\rr^{n,m,a,v}v$ alors que bien typée
  \medskip
  
  Il est suffisant :
  \begin{itemize}
  \item Pour la substitution la plus générale $\tau$ permettant de typer
    $\tau l$ dans $\Gamma$,

    Si $\Gamma,\Delta\vdash \tau l:T$ alors $\Gamma,\Delta\vdash \tau
    r : T$
  \end{itemize}
  Pour calculer $\tau$, besoin de savoir quels symboles sont injectifs
  \begin{itemize}
  \item \lstinline!def!
  \end{itemize}
\end{frame}

\section{Logiques des prédicats}

\begin{frame}[fragile]
  \frametitle{Logique des prédicats minimale}

 $$
  \begin{array}{r@{~~~::=~~~}l}
    t & x~|~f(t,\ldots,t)\\
  A&P(t_1, \ldots, t_n)~|~A \Rightarrow
  A~|~\fa_{s_1} x~A~|~\cdots~|~\fa_{s_n} x~A
  \end{array}
  $$
  \pause

\begin{lstlisting}
s1 : Type.
s2 : Type.
f : s1 -> s2.
P : s2 -> s1 -> Type.
\end{lstlisting}
  
\begin{align*}
\|P(t_1,\ldots,t_n)\| &= (\texttt{P}~|t_1|~\ldots~|t_n|)\\
\|A \Rightarrow B\| &= \|A\| \texttt{->} \|B\|\\
\|\fa_s x~A\| &= \texttt{x : s ->} \|A\|
\end{align*}

\end{frame}

\begin{frame}[fragile]
  \frametitle{Logique des prédicats constructive}
 $$
  \begin{array}{r@{~~~::=~~~}l}
    t & x~|~f(t,\ldots,t)\\
  A&P(t_1, \ldots, t_n)~|~A \Rightarrow
  A~|~\fa_s x~A\alert<1>{~|~\top~|~\bot~|~\neg A~|~A \wedge A~|~A \vee A~|~\ex_s x~A}
  \end{array}
  $$
  \pause
  Étendre le $\lambda\Pi$-calcul modulo théorie
  \begin{itemize}
  \item avec de nouveaux connecteurs
    \item avec des inductifs ?\pause
  \end{itemize}
  Non, on peut rester avec le même base
  \bigskip
  
  Idée :
  \begin{itemize}
  \item Plongement profond dans un type \lst{o}
  \item Remontée superficielle avec opérateur \lst{eps}

    en utilisant encodage imprédicatif des connecteurs
  \end{itemize}
\end{frame}

\begin{frame}[fragile]
  \frametitle{Encodage imprédicatif}
\begin{lstlisting}
o : Type.
bot : o.
imp : o -> o -> o.
or : o -> o -> o.

def eps : o -> Type.

[a,b] eps (imp a b) --> eps a -> eps b.œ\pauseœ
[] eps bot --> z : o -> eps z.œ\pauseœ
[a,b] eps (or a b) -->
  z : o -> (eps a -> eps z) -> (eps b -> eps z) -> eps z.
\end{lstlisting}


\end{frame}

\begin{frame}
\frametitle{iProverModulo}
iProver~[Korovin~2008] 
\begin{itemize}
\item prouveur automatique pour la logique des prédicats classique
\item basé (entre autres) sur la résolution
\item écrit en OCaml
\end{itemize}
\medskip

patché pour intégrer la déduction modulo théorie [Burel 2011]
\medskip

capable de sortir des preuves Dedukti pour la partie résolution [Burel 2013]

(qui est constructive !)
\end{frame}


\begin{frame}[fragile]
  \frametitle{Logique des prédicats classique}
  Idée : introduire des connecteurs différents
  
 $$
  \begin{array}{r@{~~~::=~~~}l}
    t & x~|~f(t,\ldots,t)\\
    a & P(t,\ldots,t)\\
  A&\only<2->{\ato}a~|~A \Rightarrow
  A~|~\fa x~A~|~\top~|~\bot~|~\neg A~|~A \wedge A~|~A \vee A~|~\ex
  x~A~|~\\
  \multicolumn{1}{c}{}&\only<2->{\ato_c}a~|~A \Rightarrow_c
  A~|~\fa_c x~A~|~\top_c~|~\bot_c~|~\neg_c A~|~A \wedge_c A~|~A \vee_c A~|~\ex_c x~A
  \end{array}
  $$
  
  \onslide<3->
  Définir les connecteurs classiques en fonction des intuitionnistes
  en utilisant une $\neg\neg$-traduction.

  \begin{align*}
    \ato_c a&:= \neg(\neg\ato a)\\
    a\vee_c b&:= \neg(\neg(a\vee b))\\
    \dots
  \end{align*}
\end{frame}

\begin{frame}
  \frametitle{Zenon Modulo}
  Extension de Zenon pour traiter la déduction modulo théorie
  [Halmagrand~2013]
  \bigskip
  
  Capable de produire des preuves Dedukti
  \medskip
  
  600MB de preuves gzippées (à partir d'obligation de preuves B)
\end{frame}

\section{Langages de programmation}
\begin{frame}
  \frametitle{$\lambda$-calcul non typé}

  $$app~(lam~f)~t \rr^{f,t} f~t$$
\end{frame}

\begin{frame}[fragile]
  \frametitle{$\lambda$-calcul simplement typé}
  Plongement profond des types simples
\begin{lstlisting}
type : Type.
o : type.
b : type.
arrow : type -> type -> type.
\end{lstlisting}
  
Remontée superficielle avec \lst{term}

\begin{lstlisting}
def term : type -> Type.
[a,b] term (arrow a b) --> term a -> term b.
\end{lstlisting}
\end{frame}

\begin{frame}[fragile]
  \frametitle{ML}
  Utilisation de destructeurs à la place du filtrage de motifs

$$\begin{array}{c@{~}c@{~}c@{~}c@{~}ccll}
  \texttt{{destr}}_{\texttt{{C}}} & \texttt{{R}} & \texttt{{(C t)}}  & \texttt{{f}} & \texttt{{d}} & \texttt{{-{}->}} & \texttt{{f t}} & \\
  \texttt{{destr}}_{\texttt{{C}}} & \texttt{{R}} & \texttt{{(C' t')}}& \texttt{{f}} & \texttt{{d}} & \texttt{{-{}->}} & \texttt{{d}} &
  \text{(for all other constructors \texttt{C'} for type $\tau'$)} % L'argument de C' n'est pas nécessairement de type \tau
  \end{array}$$

  Blocage de la récursion:

    \lst{@ : A : type -> B: type -> ((eps A -> eps B)} \lst{-> eps A -> eps B)}

    \lst{[R, f, t] @} $\tau'$ \lst{R f (C t) --> f (C t)}

  
\end{frame}

\begin{frame}
  \frametitle{$\varsigma$-calcul et FoCaLiZe}

  $\varsigma$-calcul : langage pour la programmation orientée objet
  \bigskip
  
  FoCaLiZe : environnement pour le développement de programmes
  certifiés
  \begin{itemize}
  \item ML comme langage d'implémentation
  \item logique des prédicats classique comme langage de spécification
  \item aspects orientés objet pour la modularité
  \end{itemize}
\end{frame}

\section{Logique d'ordre supérieur}

\begin{frame}[fragile]
\frametitle{Théorie simple des types}
Langage : 
\begin{itemize}
\item $\lambda$-calcul simplement typé
\item
  types de base
  \lst{bool} et \lst{ind}
\item constantes \lst{imp : bool -> bool -> bool} et
  \texttt{forall$_A$ : ($A$ -> bool) -> bool}
\end{itemize}
\bigskip

Plongement profond des types, des termes de type bool

\medskip
Deux fonctions pour remonter
\pause\bigskip

En fait, tout est défini à l'aide de $=$

\end{frame}

\section{Calcul des constructions inductives}

\begin{frame}
  \frametitle{Calcul des constructions}
  Deux traductions des termes :
  \begin{itemize}
  \item profonde, vus comme des termes $|t|$
  \item superficielle, vus comme des types $||t||$
  \end{itemize}

  plusieurs fonctions de remontée en fonction de la sorte

  $\epsilon_{Type}$ et $\epsilon_{Kind}$
\end{frame}

\begin{frame}[fragile]
  \frametitle{Exemple}
\begin{alltt}
\color{blue}\textbf{Definition} id_Type := \textbf{fun} x : Type => x.
\end{alltt}

est traduit par
\begin{overprint}
\onslide<1>
\begin{alltt}
id_Type : \tytrans{\ccprop \rightarrow \ccprop}.  
[] id_Type --> \tetrans{\lambda x : \ccprop. x}.
\end{alltt}
\onslide<2>
\begin{alltt}
id_Type : \tytrans{\ccprop \rightarrow \ccprop}.
[] id_Type --> \alert{x : \tytrans{\ccprop} => x}.
\end{alltt}
\onslide<3>
\begin{alltt}
id_Type : \tytrans{\ccprop \rightarrow \ccprop}.
[] id_Type --> x : \alert{U_Type} => x.
\end{alltt}
\onslide<4>
\begin{alltt}
id_Type : \alert{e_Kind \tetrans{\ccprop \rightarrow \ccprop}}.
[] id_Type --> x : U_Type => x.
\end{alltt}
\onslide<5>
\begin{alltt}
id_Type : e_Kind \alert{\tetrans{\Pi _ : \ccprop. \ccprop}}.
[] id_Type --> x : U_Type => x.
\end{alltt}
\onslide<6>
\begin{alltt}
id_Type : e_Kind \alert{(pi_KK \tetrans{\ccprop} (fun _ : \tytrans{\ccprop} => \tetrans{\ccprop}))}.
[] id_Type --> x : U_Type => x.
\end{alltt}
\onslide<7>
\begin{alltt}
id_Type : \alert{w : e_Kind \tetrans{\ccprop}-> e_Kind((fun _ : \tytrans{\ccprop} => \tetrans{\ccprop}) w)}.
[] id_Type --> x : U_Type => x.
\end{alltt}
\onslide<8>
\begin{alltt}
id_Type : w : e_Kind \tetrans{\ccprop} -> \alert{e_Kind \tetrans{\ccprop}}.
[] id_Type --> x : U_Type => x.
\end{alltt}
\onslide<9>
\begin{alltt}
id_Type : \alert{w : e_Kind u_Type -> e_Kind u_Type}.
[] id_Type --> x : U_Type => x.
\end{alltt}
\onslide<10>
\begin{alltt}
id_Type : \alert{w : U_Type -> U_Type}.
[] id_Type --> x : U_Type => x.
\end{alltt}
\end{overprint}\bigskip

\begin{itemize}
\item<2-> $\lambda$s traduits comme des $\lambda$s 
\item<3-> à chaque sorte \sort{s} correspond un univers \verb;U_;\sort s
 \parbox{2cm}{\onslide<9->\parbox{5cm}{et une constante \texttt{u\_}\sort s}}\onslide<3->
\item<4-> \verb;e_;\sort s: fonction de remontée pour \sort s
\item<6-> \verb;pi_KK;: plongement profond des produits dépendants
\item<7-> règles de réécriture pour remonter

\verb;e_Kind (pi_KK x y) --> w : e_Kind x -> e_Kind (y w).;

\onslide<10->\verb;e_Kind u_Type --> U_Type.;
\end{itemize}
\end{frame}

\begin{frame}[fragile]
  \frametitle{Inductifs}

  Cf. ML
  \bigskip

  Utilisation d'éliminateur au lieu de destructeurs + récursion

\begin{lstlisting}
elim_list : A : Type
  -> P : (List A -> Type)
  -> P (nil A)
  -> (x : A -> l : list A -> P l -> P (cons A x l))
  -> l : list A
  -> P l.
\end{lstlisting}
  
\end{frame}

\begin{frame}[fragile]
  \frametitle{Hiérarchie d'univers}
  Univers indexés par un niveau

  \begin{itemize}
  \item $Type_i : Type_{i+1}$

  \item $Type_i\subset Type_{i+1}$
  \end{itemize}
  \bigskip
  
  Conversion explicite avec $\uparrow_i : Type_i \to Type_{i+1}$
  \medskip
  
  Plus d'unicité du typage
  \begin{itemize}
  \item $\uparrow_i \Pi_i x:A.~B$\qquad vs.\qquad $\Pi_{i+1}
    x : \uparrow_i A.~\uparrow_i B$
  \end{itemize}
\begin{lstlisting}
[i, a, b] pi _ (lift i a) (x => lift {i} (b x)) -->
          lift i (pi i a (x => b x)).
\end{lstlisting}
  
\end{frame}

\end{document}

%%% Local Variables:
%%%   mode: latex 
%%%   mode: flyspell
%%%   ispell-local-dictionary: "francais"
%%% End: 
